= 
set was used to perform a simulation in three postures (extension MCP = 10°, PIP = 10°, DIP = 10°,
mid-flexion MCP = 45°, PIP = 45°, DIP = 10°, and flexion MCP = 90°, PIP = 90°, DIP = 80°) and for
two loading conditions: all muscle forces (UI, EDC, RI, and LU) were set to Φ = 2.9 N and Φ =
=
extensor lateral fibers. The first row (pannels a and b) corresponds to extension (MCP = 10°, PIP = 10°, DIP = 10°), the second
row corresponds to mid-flexion (MCP = 45°, PIP = 45°, DIP = 10°, pannels c and d), and the third row corresponds to flexion
(MCP = 90°, PIP = 90°, DIP = 80°, pannels e and f). Two loading conditions are shown by bars of light (Φ = 2.9 N) and dark
(Φ = 5.9 N in UI, EDC, RI, and LU muscles) colors, respectively. Within each loading condition, the same force magnitude was
both loading conditions (2.9 N loading: F(2,708)=32.56, p<0.001, 5.9 N loading: F(2,708)=129.84,
p<0.001). There was a significant posture × fiber bundle interaction effect (2.9 N loading F(6,708)=13.54,
p<0.001), 5.9 N loading F(6,708)=15.52) suggesting that the influence of posture varies according the
collumn corresponds to a trivial model. The first row corresponds to extension  (MCP = 10°; PIP = 10°; DIP = 10°), the second
row coresponds to mid-flexion (MCP = 45°; PIP = 45°; DIP = 10°), and the third row corresponds to flexion posture (MCP =
90°; PIP = 90°; DIP = 80°). The full-loading state, which corresponds to loading of the extensor mechanism models by all four
Φ=2.9/5.9 N, is shown by a circle in each panel (“full loading”).
mechanism model, right collumn corresponds to a trivial model. First row corresponds to  extension posture (MCP = 10°;
PIP ;= 10°; DIP = 10°), second row coresponds to mid-flexion posture (MCP = 45°; PIP = 45°; DIP = 10°), third row
corresponds to flexion posture (MCP = 90°; PIP = 90°; DIP = 80°). Dark area cooresponds to 5.9 N loading and light area
